\documentclass[UTF8,a4paper,12pt]{ctexart}

\usepackage[a4paper,margin=2.5cm]{geometry}
\usepackage{amsmath,amssymb}
\usepackage{booktabs}
\usepackage{graphicx}
\usepackage{siunitx}
\usepackage{hyperref}
\usepackage{pgfplots}
\usepackage{pgfplotstable}
\usepackage{longtable}

\pgfplotsset{compat=1.18}
\usepgfplotslibrary{statistics}
\usepgfplotslibrary{groupplots}

\title{基于日步数与睡眠时长的个人行为数据统计分析报告}
\author{姓名:何子宁\quad 学号:524030910131}
\date{2025年12月}

\begin{document}
\maketitle

\begin{abstract}
本文基于运动手环记录的90天日度数据(步数与睡眠时长)进行统计分析。首先给出两变量的描述性统计与可视化(时间序列、分布图),随后对步数采用对数正态近似、对睡眠时长采用正态近似,完成参数估计与95\%置信区间,并以“周末与工作日步数是否不同”“步数与睡眠是否相关”为例给出假设检验过程,最后报告线性回归结果。样本中日步数均值为8500.93步、波动较大,睡眠时长均值为7.26小时;相关性检验得到$r=0.518$($p\approx1.71\times10^{-7}$),回归中睡眠时长的斜率估计为正(约1775步/小时)。结论主要用于说明方法应用,结果仍可能受极端值与其他未纳入因素影响。
\end{abstract}

\section{引言}
日步数与睡眠时长是个人健康管理中常用的两个日度指标:步数反映活动水平,睡眠时长反映恢复与作息。以90天的日度记录为样本,可以直观看到波动与分布特点,也能按照课堂方法把概率建模、估计、检验与回归的步骤完整做一遍。

本报告聚焦以下问题:
\begin{enumerate}
  \item 日步数与睡眠时长的总体水平与离散程度如何?是否存在异常值或偏态?
  \item 两变量在90天内的日度波动与趋势如何?是否存在明显的阶段性变化?
  \item 日步数与睡眠时长是否存在统计相关?睡眠时长对步数的线性关联强度有多大?
\end{enumerate}

\section{数据集基本介绍}
\subsection{数据来源}
数据来自运动手环记录的每日汇总值,按日口径整理得到90天的日步数与睡眠时长序列(起始日期为2025-09-16,截止日期为2025-12-14)。

\subsection{变量与类型}
\begin{itemize}
  \item 日期:$t$(按日时间索引)
  \item 日步数:$Y_t$(离散、非负计数型)
  \item 睡眠时长:$S_t$(连续型,单位:小时)
\end{itemize}

为辅助描述时间结构与趋势,可由日期推导衍生变量:
\begin{itemize}
  \item 是否周末:$W_t\in\{0,1\}$(周末为1,工作日为0)
  \item 7日平均步数/睡眠时间:$\overline{Y}_t=\frac{1}{7}\sum_{i=0}^{6}Y_{t-i}$,$\overline{S}_t=\frac{1}{7}\sum_{i=0}^{6}S_{t-i}$
\end{itemize}

\subsection{数据规模}
原始核心变量维数为2($Y_t,S_t$),样本量为$n=90$。用于可视化与检验的$W_t,\overline{Y}_t,\overline{S}_t$为由日期或滑动窗口计算得到的衍生变量。

\subsection{数据表结构}
表\ref{tab:data}展示数据表结构与样例记录;完整90天数据见附录表\ref{tab:full_all}。

\begin{table}[h]
\centering
\sisetup{detect-all}
\begin{tabular}{lllll}
\toprule
日期 & 日步数$Y_t$ & 睡眠时长$S_t$ (h) & 周末$W_t$ & 备注 \\
\midrule
2025-09-16 & 812 & 2.3 & 0 &  \\
2025-09-17 & 1247 & 6.4 & 0 &  \\
2025-09-18 & 1793 & 4.1 & 0 &  \\
\bottomrule
\end{tabular}
\caption{按日数据记录格式与样例}
\label{tab:data}
\end{table}

\subsection{数据预处理}
\begin{itemize}
  \item 测量范围统一:步数与睡眠均为按日汇总值,日期连续覆盖90天。
  \item 缺失值:本样本两变量均无缺失。
  \item 异常值:步数存在极低值(最低812步),睡眠也存在极端值(最低2.3小时、最高13.2小时)。分析过程中保留原始记录,并在结论中讨论其可能含义与对统计分析的影响。
\end{itemize}

\subsection{衍生变量}
为刻画短期噪声与趋势,计算步数与睡眠的7日移动平均,并在图中与原始序列叠加展示。

\section{可视化分析}
\subsection{总体描述}
表\ref{tab:desc}给出两变量的描述性统计。均值与标准差反映总体水平与离散程度;考虑到存在极端值,同时报告中位数等稳健统计量。样本期内:
\begin{itemize}
  \item 日步数均值为8500.93步、标准差为5421.00步,中位数为6653步;最小值为812步(2025-09-16),最大值为21836步(2025-11-15)。
  \item 睡眠时长均值为7.26小时、标准差为1.58小时,中位数为7.2小时;最小值为2.3小时(2025-09-16),最大值为13.2小时(2025-11-15)。
\end{itemize}

\begin{table}[h]
\centering
\begin{tabular}{lrrrrr}
\toprule
变量 & 均值 & 标准差 & 最小值 & 中位数 & 最大值 \\
\midrule
日步数$Y_t$ & 8500.93 & 5421.00 & 812 & 6653 & 21836 \\
睡眠时长$S_t$ & 7.26 & 1.58 & 2.3 & 7.2 & 13.2 \\
\bottomrule
\end{tabular}
\caption{描述性统计汇总}
\label{tab:desc}
\end{table}

\subsection{时间序列图与周期性}
图\ref{fig:ts_both}展示日步数与睡眠时长的时间序列及各自的7日移动平均。从图中可见步数的波动幅度明显大于睡眠时长;叠加移动平均后,部分时段两者变化方向较一致。

\pgfplotstableread[col sep=space]{%
day steps sleep weekend ma7_steps ma7_sleep
1 812 2.3 0 nan nan
2 1247 6.4 0 nan nan
3 1793 4.1 0 nan nan
4 963 7.9 0 nan nan
5 2218 5.3 1 nan nan
6 3076 6.7 1 nan nan
7 15684 8.6 0 3684.71 5.90
8 2684 7.2 0 3952.14 6.60
9 4189 6.1 0 4372.43 6.56
10 3827 8.0 0 4663.00 7.11
11 5126 6.6 0 5257.71 6.93
12 4573 7.8 1 5594.14 7.29
13 2439 4.0 1 5503.14 6.90
14 5297 7.1 0 4019.29 6.69
15 6148 6.7 0 4514.14 6.61
16 4861 8.7 0 4610.14 6.99
17 3472 5.1 0 4559.43 6.57
18 5924 7.5 0 4673.43 6.70
19 18946 9.6 1 6726.71 6.96
20 5638 6.3 1 7183.71 7.29
21 6152 7.3 0 7305.86 7.31
22 6217 6.2 0 7315.71 7.24
23 4284 8.9 0 7233.29 7.27
24 2796 4.8 0 7136.71 7.23
25 5071 7.8 0 7014.86 7.27
26 3928 6.5 1 4869.43 6.83
27 3346 8.4 1 4542.00 7.13
28 5763 7.1 0 4486.43 7.10
29 4479 5.7 0 4238.14 7.03
30 6018 7.5 0 4485.86 6.83
31 6386 6.1 0 4998.71 7.01
32 7193 8.0 0 5301.86 7.04
33 6827 7.3 1 5716.00 7.16
34 17891 8.8 1 7793.86 7.21
35 8234 8.8 0 8146.86 7.46
36 7049 5.9 0 8514.00 7.49
37 9037 7.7 0 8945.29 7.51
38 8472 9.1 0 9243.29 7.94
39 7816 6.2 0 9332.29 7.69
40 9483 7.9 1 9711.71 7.77
41 10137 6.6 1 8604.00 7.46
42 10984 8.5 0 8996.86 7.41
43 9826 6.0 0 9393.57 7.43
44 11973 7.0 0 9813.00 7.33
45 13482 9.7 0 10528.71 7.41
46 12461 6.4 0 11192.29 7.44
47 13927 8.6 1 11827.14 7.54
48 20861 9.8 1 13359.14 8.00
49 15863 10.0 0 14056.14 8.21
50 14472 6.9 0 14719.86 8.34
51 16834 9.5 0 15414.29 8.70
52 15429 7.1 0 15692.43 8.33
53 5094 6.3 0 14640.00 8.31
54 16372 6.3 1 14989.29 7.99
55 5871 8.0 1 12847.86 7.73
56 17418 7.2 0 13070.00 7.33
57 5587 7.2 0 11800.71 7.37
58 18394 6.7 0 12023.57 6.97
59 6097 6.7 0 10690.43 6.91
60 6924 6.6 0 10951.86 6.96
61 21836 13.2 1 11732.43 7.94
62 20317 6.1 1 13796.14 7.67
63 15924 8.3 0 13582.71 7.83
64 9874 6.8 0 14195.14 7.77
65 7218 7.6 0 12598.57 7.90
66 4837 8.5 0 12418.57 8.16
67 2986 5.0 0 11856.00 7.93
68 5183 7.3 1 9477.00 7.09
69 6097 6.7 1 7445.57 7.17
70 8726 8.7 0 6417.29 7.23
71 9184 6.1 0 6318.71 7.13
72 10493 9.0 0 6786.57 7.33
73 6479 7.5 0 7021.14 7.19
74 4026 5.9 0 7169.71 7.31
75 2284 6.8 1 6755.57 7.24
76 14962 6.2 1 8022.00 7.17
77 17384 7.7 0 9258.86 7.03
78 827 3.1 0 8065.00 6.60
79 3379 9.1 0 7048.71 6.61
80 6924 6.6 0 7112.29 6.49
81 8793 7.4 0 7793.29 6.70
82 4686 8.2 1 8136.43 6.90
83 12473 8.7 1 7780.86 7.26
84 4327 8.0 0 5915.57 7.30
85 5094 6.3 0 6525.14 7.76
86 14918 7.3 0 8173.57 7.50
87 7634 7.9 0 8275.00 7.69
88 9781 5.8 0 8416.14 7.46
89 15500 8.4 1 9961.00 7.49
90 19573 9.9 1 10975.29 7.66
}\datatable

\begin{figure}[h]
\centering
\begin{tikzpicture}
\begin{groupplot}[
  group style={group size=1 by 2, vertical sep=1cm},
  width=\linewidth,
  height=5.5cm,
  xlabel=天序号,
  grid=major
]
\nextgroupplot[ylabel=步数, use fpu=true, legend pos=north west]
\addplot+[mark=none, blue] table[x=day, y=steps] {\datatable};
\addplot+[mark=none, red, thick] table[x=day, y=ma7_steps] {\datatable};
\legend{步数,7日移动平均}
\nextgroupplot[ylabel=睡眠时长(h), use fpu=true, legend pos=north west]
\addplot+[mark=none, blue] table[x=day, y=sleep] {\datatable};
\addplot+[mark=none, red, thick] table[x=day, y=ma7_sleep] {\datatable};
\legend{睡眠,7日移动平均}
\end{groupplot}
\end{tikzpicture}
\caption{日步数与睡眠时长的时间序列及7日移动平均}
\label{fig:ts_both}
\end{figure}

\begin{figure}[h]
\centering
\begin{tikzpicture}
\begin{groupplot}[
  group style={group size=1 by 2, vertical sep=1cm},
  width=\linewidth,
  height=5.5cm,
  ylabel=频数,
  ybar interval,
  grid=major
]
\nextgroupplot[xlabel=步数]
\addplot+[hist={bins=15}] table[y=steps] {\datatable};
\nextgroupplot[xlabel=睡眠时长(h)]
\addplot+[hist={bins=12}] table[y=sleep] {\datatable};
\end{groupplot}
\end{tikzpicture}
\caption{日步数与睡眠时长的分布直方图}
\label{fig:dist_both}
\end{figure}

\section{数据建模}
\subsection{(a) 概率模型、参数估计与区间估计}
考虑到步数为非负且分布右偏明显,本文用对数正态分布对步数作近似;睡眠时长为连续变量,采用正态分布近似,便于进行参数估计与区间估计。
\begin{itemize}
  \item 日步数$Y_t$为非负且右偏明显,采用对数正态模型近似:$Y_t\sim \mathrm{LogNormal}(\mu_Y,\sigma_Y^2)$,等价于$Z_t=\ln Y_t\sim \mathcal{N}(\mu_Y,\sigma_Y^2)$。
  \item 睡眠时长$S_t$为连续变量,采用正态模型:$S_t\sim \mathcal{N}(\mu_S,\sigma_S^2)$。
\end{itemize}

在独立同分布假设下,对数正态模型的极大似然估计为
\[
\hat\mu_Y=\frac{1}{n}\sum_{t=1}^n \ln Y_t,\qquad
\hat\sigma_Y^2=\frac{1}{n}\sum_{t=1}^n(\ln Y_t-\hat\mu_Y)^2.
\]
由样本计算得到$\hat\mu_Y=8.817$、$\hat\sigma_Y=0.733$;对应的模型隐含步数均值为$\exp(\hat\mu_Y+\tfrac12\hat\sigma_Y^2)=8833.32$(步)。对$Z_t$做区间估计得到
\[
\mu_Y\in[8.666,\,8.969],\qquad
\sigma_Y^2\in[0.410,\,0.737].
\]

在正态模型$S_t\sim\mathcal{N}(\mu_S,\sigma_S^2)$下,$\mu_S$可用样本均值$\bar S$估计,$\sigma_S^2$可由样本方差$s_S^2$刻画。样本计算得到$\bar S=7.258$(h),$s_S=1.581$(h)。进一步由$t$分布与$\chi^2$分布可构造95\%置信区间:
\[
\mu_S\in[6.927,\,7.589],\qquad
\sigma_S^2\in[1.903,\,3.436].
\]

\subsection{(b) 二维散点图与线性回归}
取二维随机变量$(X_t,Y_t)=(S_t,Y_t)$。图\ref{fig:scatter}展示散点图与最小二乘拟合直线。
\begin{figure}[h]
\centering
\begin{tikzpicture}
\begin{axis}[
  width=\linewidth,
  height=6cm,
  xlabel=睡眠时长(小时),
  ylabel=步数,
  use fpu=true,
  grid=major
]
\addplot+[only marks, mark=*, mark size=1.2pt, blue] table[x=sleep, y=steps] {\datatable};
\addplot+[no marks, red, thick, domain=2.3:13.2, samples=2] {-4385.06 + 1775.47*x};
\end{axis}
\end{tikzpicture}
\caption{日步数与睡眠时长散点图}
\label{fig:scatter}
\end{figure}

建立线性回归模型
\[
Y_t=\beta_0+\beta_1 X_t+\varepsilon_t,\quad \mathbb{E}(\varepsilon_t\mid X_t)=0,
\]
并进一步加入周末指示$W_t$作对比:
\[
Y_t=\beta_0+\beta_1 X_t+\beta_2 W_t+\varepsilon_t.
\]
表\ref{tab:reg}给出估计结果。模型(1)的$R^2=0.268$,表明线性模型可解释约26.8\%的步数波动;睡眠时长系数为正且显著。加入周末控制后,睡眠时长系数仍为正且显著,而周末项不显著。
\begin{table}[h]
\centering
\small
\begin{tabular}{lrrr|rrr}
\toprule
 & \multicolumn{3}{c}{模型(1):$Y\sim S$} & \multicolumn{3}{c}{模型(2):$Y\sim S+W$} \\
\cmidrule(lr){2-4}\cmidrule(lr){5-7}
变量 & 系数估计 & 标准误差 & $p$值 & 系数估计 & 标准误差 & $p$值 \\
\midrule
截距$\beta_0$ & -4385.06 & 2321.18 & 0.062 \\
睡眠时长$\beta_1$ & 1775.47 & 312.57 & $1.71\times 10^{-7}$ & 1699.47 & 315.05 & $5.85\times 10^{-7}$ \\
周末指示$\beta_2$ &  &  &  & 1581.44 & 1093.18 & 0.152 \\
\bottomrule
\end{tabular}
\caption{线性回归结果(因变量:日步数)}
\label{tab:reg}
\end{table}

\section{假设检验}
\subsection{检验1:周末与工作日步数差异}
为检验步数是否存在显著的“周末效应”,将样本按$W_t$分为工作日组与周末组,比较两组步数均值是否相同。
\[
H_0:\mu_{\text{weekend}}=\mu_{\text{weekday}},\quad
H_1:\mu_{\text{weekend}}\ne\mu_{\text{weekday}}
\]
Welch两样本$t$检验的统计量可写为
\[
T=\frac{\bar Y_{\text{weekend}}-\bar Y_{\text{weekday}}}{\sqrt{s^2_{\text{weekend}}/n_{\text{weekend}}+s^2_{\text{weekday}}/n_{\text{weekday}}}},
\]
其在$H_0$下近似服从自由度为Welch--Satterthwaite公式给出的$t$分布;在显著性水平$\alpha=0.05$下,拒绝域为$|T|>t_{0.975,\mathrm{df}}$。同时报告Mann--Whitney秩和检验(其统计量经标准化后在大样本下近似服从$\mathcal{N}(0,1)$,拒绝域为$|Z|>1.96$)。

\subsection{结果呈现}
表\ref{tab:weekday_weekend}与表\ref{tab:test}给出两组的描述统计与检验结果。周末步数均值高于工作日(10324.77 vs 7760.00),但在常用显著性水平下差异不显著。

\begin{table}[h]
\centering
\begin{tabular}{lrrrr}
\toprule
组别 & 样本量 & 均值 & 标准差 & 中位数 \\
\midrule
工作日 & 64 & 7760.00 & 4598.54 & 6432.5 \\
周末 & 26 & 10324.77 & 6816.48 & 8155 \\
\bottomrule
\end{tabular}
\caption{工作日/周末的步数统计}
\label{tab:weekday_weekend}
\end{table}

\begin{table}[h]
\centering
\begin{tabular}{lrr}
\toprule
检验方法 & 统计量 & $p$值 \\
\midrule
两样本$t$检验(Welch) & $t=1.763$(df=34.63) & 0.087 \\
秩和检验(Mann--Whitney) & $z=-1.166$ & 0.244 \\
\bottomrule
\end{tabular}
\caption{组间差异检验结果}
\label{tab:test}
\end{table}

\subsection{检验2:步数与睡眠时长的相关性}
将“是否存在线性相关”表述为对总体相关系数的检验:
\[
H_0:\rho=0,\quad H_1:\rho\ne 0.
\]
在$H_0$下,检验统计量
\[
T=\frac{r\sqrt{n-2}}{\sqrt{1-r^2}}
\]
服从自由度为$n-2$的$t$分布。给定显著性水平$\alpha=0.05$,拒绝域为$|T|>t_{0.975,n-2}$;也可使用$p$值规则。由样本计算得到$r=0.518$(95\% CI:$[0.348,0.655]$),对应$t=5.680$(df=88),$p\approx1.71\times10^{-7}$,据此拒绝$H_0$。若检验不显著,则说明在该样本下没有足够证据支持线性相关。散点图见图\ref{fig:scatter}。

\subsubsection{结果呈现}
\begin{table}[h]
\centering
\begin{tabular}{ccccc}
\toprule
$r$ & 95\% CI & $t$(df) & $p$值 & 结论($\alpha=0.05$) \\
\midrule
0.518 & $[0.348,0.655]$ & 5.680(88) & $1.71\times 10^{-7}$ & 拒绝$H_0$ \\
\bottomrule
\end{tabular}
\caption{相关性检验结果($H_0:\rho=0$)}
\label{tab:corr_test}
\end{table}

\section{结论}
\begin{itemize}
  \item 本文依次完成数据集介绍、可视化、概率建模与参数估计、区间估计、假设检验与线性回归,并给出对应的计算结果与解释。
  \item 在2025-09-16至2025-12-14的90天样本中,日步数均值为8500.93步、标准差为5421.00步,中位数为6653步,存在极低步数日(812步);睡眠时长均值为7.26小时、标准差为1.58小时,中位数为7.2小时,亦存在极端睡眠日(2.3小时与13.2小时)。
  \item 两变量关系方面,日步数与睡眠时长呈中等强度正相关(Pearson $r=0.518$,95\% CI:$[0.348,0.655]$;Spearman $\rho=0.409$)。线性回归结果显示:睡眠时长每增加1小时,日步数平均增加约1775步,且在统计意义上显著。
  \item 补充分析表明周末步数均值高于工作日,但差异未达到常用显著性水平;在回归中加入周末控制后,睡眠对步数的正向关联仍显著,而周末项不显著。
  \item 本文只基于两个核心变量做分析,且数据为单人记录;步数与睡眠都容易受当天安排、设备记录等影响,因此结果更适合作为相关性与方法示例,而非因果结论。
\end{itemize}

\section{参考文献}
\begin{thebibliography}{9}
\bibitem{intro_stat}
卫淑芝, 熊德文, 皮玲. 概率论与数理统计[M].高等教育出版社, 2020
\end{thebibliography}
\appendix
\section{附录:完整数据}
\footnotesize
\pgfplotstabletypeset[
  begin table=\begin{longtable},
  end table=\end{longtable},
  columns={day,steps,sleep,weekend},
  columns/day/.style={column name=天序号},
  columns/steps/.style={column name=步数, fixed, precision=0},
  columns/sleep/.style={column name=睡眠时长(h), fixed, precision=1},
  columns/weekend/.style={column name=周末, fixed, precision=0},
  every head row/.style={
    before row=\caption{完整数据(90天)}\label{tab:full_all}\\\toprule,
    after row=\midrule\endfirsthead
              \caption[]{完整数据(90天,续)}\\\toprule
              天序号 & 步数 & 睡眠时长(h) & 周末\\
              \midrule\endhead
              \midrule\multicolumn{4}{r}{续下页}\\\endfoot
              \bottomrule\endlastfoot
  },
]{\datatable}




\end{document}
