\documentclass[UTF8,a4paper,12pt]{ctexart}
\usepackage[a4paper,margin=2.5cm]{geometry}
\usepackage{amsmath,amssymb}
\usepackage{graphicx}
\usepackage{circuitikz} % 用于绘制电路图
\usepackage{titlesec}
\usepackage{fancyhdr}
\usepackage{booktabs}

% 设置页面风格
\pagestyle{fancy}
\fancyhf{}
\lhead{电路理论自编习题}
\rhead{非线性电路与混沌}
\cfoot{\thepage}

\title{\textbf{基于非线性负阻元件的蔡氏电路动力学分析}}
\author{何子宁}
\date{Jan. 2026}

\begin{document}

\maketitle

\section{问题来源}

本题灵感来源于本学期的《大学物理实验》课程。在“非线性电路混沌现象”实验中,我实际搭建了蔡氏电路并观察到了双涡卷吸引子。结合《电路理论》课程中关于非线性电阻和动态电路分析的内容,我将实验电路抽象为如下理论模型,并尝试从电路原理角度求解其静态工作点,以解释电路起振的初始条件。

\section{自编题目}

\textbf{【题目描述】} 

图 \ref{fig:chua_circuit} 所示为蔡氏电路(Chua's Circuit)的简化模型。电路包含四个线性动态/静态元件:电感 $L$、电阻 $R$、电容 $C_1$ 和 $C_2$。最右侧的元件 $N_R$ 是一个非线性有源电阻(蔡氏二极管),其伏安特性曲线 $i_R = g(v_R)$ 如图 \ref{fig:nr_iv} 所示。

\begin{figure}[h]
    \centering
    \begin{circuitikz}[american, scale=1.0]
        % 绘制电路主体
        \draw (0,0) coordinate(GND) to[C, l=$C_2$, v=$v_{C2}$] (0,3) coordinate(N2);
        \draw (0,0) -- (6,0);
        \draw (N2) to[L, l=$L$, i=$i_L$] (3,3) coordinate(N1);
        \draw (3,0) to[R, l=$R$, i<=$i_R$] (3,3); % 这里R实际上是串联在C1和C2之间的,标准Chua电路通常是 R 连接 C1 C2
        % 修正标准Chua电路结构:
        % L 并联 C2, 然后串联 R, 然后并联 C1 和 NR
        % 让我们画一个标准的 Chua's Circuit 拓扑
    \end{circuitikz}
    
    % 重新绘制标准拓扑
    \begin{circuitikz}[american, scale=1.0]
        \draw (0,0) coordinate(ref) to[L, l=$L$, i=$i_L$] (0,3) coordinate(n_L);
        \draw (n_L) -- (2,3) coordinate(n_C2);
        \draw (2,0) to[C, l=$C_2$, v=$v_2$] (2,3);
        \draw (2,3) to[R, l=$R$] (5,3) coordinate(n_C1);
        \draw (5,0) to[C, l=$C_1$, v=$v_1$] (5,3);
        \draw (5,3) -- (7,3) coordinate(n_NR);
        \draw (7,0) to[R, l=$N_R$] (7,3); % 用R代替非线性电阻符号,加文字说明
        \node at (7.8, 1.5) {$g(v_1)$};
        
        \draw (0,0) -- (7,0);
        
        % 标注节点
        \node[above] at (n_C2) {节点 2};
        \node[above] at (n_C1) {节点 1};
    \end{circuitikz}
    \caption{蔡氏电路原理图}
    \label{fig:chua_circuit}
\end{figure}

非线性电阻 $N_R$ 的伏安特性 $i_R = g(v_1)$ 满足以下分段线性关系(其中 $v_1$ 为电容 $C_1$ 两端电压):
\begin{equation}
    g(v_1) = m_1 v_1 + \frac{1}{2}(m_0 - m_1)(|v_1 + E| - |v_1 - E|)
\end{equation}
其几何含义如图 \ref{fig:nr_iv} 所示:在 $|v_1| < E$ 区域斜率为 $m_0$,在 $|v_1| > E$ 区域斜率为 $m_1$。通常实验中配置 $m_0 < m_1 < 0$,即表现为负阻特性。

\begin{figure}[h]
    \centering
    \begin{tikzpicture}[scale=0.8]
        \draw[->] (-4,0) -- (4,0) node[right] {$v_1$};
        \draw[->] (0,-3) -- (0,3) node[above] {$i_R$};
        
        % 绘制分段线性曲线
        \draw[thick, blue] (-3, -1.5) -- (-1.5, -2.5) -- (1.5, 2.5) -- (3, 1.5);
        
        % 标注
        \draw[dashed] (1.5,0) node[below]{$E$} -- (1.5, 2.5);
        \draw[dashed] (-1.5,0) node[above]{$-E$} -- (-1.5, -2.5);
        \node at (0.5, 1.5) {斜率 $m_0$};
        \node at (2.5, 2.2) {斜率 $m_1$};
    \end{tikzpicture}
    \caption{非线性电阻 $N_R$ 的 $i-v$ 特性曲线}
    \label{fig:nr_iv}
\end{figure}

\textbf{问题:}
\begin{enumerate}
    \item 选取电容电压 $v_1, v_2$ 和电感电流 $i_L$ 作为状态变量,列写该电路的状态方程。
    \item 设电路参数满足 $G = 1/R$,且 $m_0 < -G < m_1 < 0$。当电路处于\textbf{直流稳态}时(即电容相当于开路,电感相当于短路),试求此时电路的\textbf{静态工作点}(即 $v_1, v_2, i_L$ 的值)。
    \item (选做)结合第四章非线性电阻的\textbf{负载线法}(Load Line Method)思想,讨论为什么在 $m_0 < -G$ 的条件下会存在非零的静态工作点。
\end{enumerate}

\newpage

\section{解答过程}

\subsection{1. 建立状态方程}
选取独立电容电压 $v_1, v_2$ 和独立电感电流 $i_L$ 为状态变量。

\textbf{(1) 对节点 1 应用 KCL:}
流出节点的电流之和为零。
\begin{equation*}
    C_1 \frac{dv_1}{dt} + \frac{v_1 - v_2}{R} + g(v_1) = 0
\end{equation*}
整理得:
\begin{equation}
    \frac{dv_1}{dt} = \frac{1}{C_1} [ G(v_2 - v_1) - g(v_1) ]
\end{equation}
其中 $G = 1/R$ 为线性电阻的电导。

\textbf{(2) 对节点 2 应用 KCL:}
\begin{equation*}
    C_2 \frac{dv_2}{dt} + \frac{v_2 - v_1}{R} + i_L = 0
\end{equation*}
整理得:
\begin{equation}
    \frac{dv_2}{dt} = \frac{1}{C_2} [ G(v_1 - v_2) - i_L ]
\end{equation}

\textbf{(3) 对电感支路应用 KVL:}
电感电压等于节点2电压(注意参考方向,图中 $i_L$ 从下往上流则 $v_L = -v_2$,若 $i_L$ 定义为从左向右流入电感,则需根据图示调整)。
假设图 \ref{fig:chua_circuit} 中 $i_L$ 方向定义为流出节点2进入地(或电感两端电压为 $v_2 - 0$),则:
\begin{equation*}
    L \frac{di_L}{dt} = -v_2 \quad (\text{假设 } i_L \text{ 方向如图中流向电感支路且电感接地})
\end{equation*}
\textit{注:通常蔡氏电路标准方程中,电感电流方向定义不同会相差一个负号。若定义 $i_L$ 为流过L的电流,且L两端电压为 $v_L = v_2$,则:}
\begin{equation}
    \frac{di_L}{dt} = -\frac{1}{L} v_2
\end{equation}

综上,电路的状态方程组为:
\begin{equation}
    \left\{
    \begin{aligned}
        \frac{dv_1}{dt} &= \frac{1}{C_1} [ G(v_2 - v_1) - g(v_1) ] \\
        \frac{dv_2}{dt} &= \frac{1}{C_2} [ G(v_1 - v_2) - i_L ] \\
        \frac{di_L}{dt} &= -\frac{1}{L} v_2
    \end{aligned}
    \right.
\end{equation}

\subsection{2. 求解静态工作点}
当电路处于直流稳态时,所有状态变量对时间的导数为零,即 $\frac{d}{dt} = 0$。
令状态方程左边为 0:
\begin{equation}
    \left\{
    \begin{aligned}
        G(v_2 - v_1) - g(v_1) &= 0 \quad \cdots (a) \\
        G(v_1 - v_2) - i_L &= 0 \quad \cdots (b) \\
        v_2 &= 0 \quad \cdots (c)
    \end{aligned}
    \right.
\end{equation}

由 (c) 式直接得:
\begin{equation}
    v_2 = 0
\end{equation}

将 $v_2=0$ 代入 (a) 式:
\begin{equation}
    G(0 - v_1) - g(v_1) = 0 \implies g(v_1) = -G v_1
\end{equation}
该式即为非线性电阻的负载线方程。我们需要求解非线性电阻特性曲线 $i = g(v_1)$ 与通过原点的直线 $i = -G v_1$ 的交点。

根据 $g(v_1)$ 的分段线性定义:
\[ g(v_1) = \begin{cases} 
m_1 v_1 + (m_0 - m_1)E & v_1 > E \\
m_0 v_1 & |v_1| \le E \\
m_1 v_1 - (m_0 - m_1)E & v_1 < -E 
\end{cases} \]

我们需要解方程 $g(v_1) + G v_1 = 0$:

\textbf{情况 1:$|v_1| \le E$}
\[ m_0 v_1 + G v_1 = 0 \implies (m_0 + G) v_1 = 0 \]
由于 $m_0 \ne -G$,故解得 $v_1 = 0$。
此时 $v_2=0$,代入 (b) 式得 $i_L = G(0-0) = 0$。
\textbf{工作点 $Q_0$:} $(0, 0, 0)$。

\textbf{情况 2:$v_1 > E$}
\[ m_1 v_1 + (m_0 - m_1)E + G v_1 = 0 \]
\[ (m_1 + G) v_1 = (m_1 - m_0)E \]
\[ v_1 = \frac{m_1 - m_0}{m_1 + G} E \]
题目给定条件 $m_0 < -G < m_1 < 0$。
\begin{itemize}
    \item 分子 $m_1 - m_0 > 0$(因为 $m_1$ 大于 $m_0$)
    \item 分母 $m_1 + G$:因为 $-G < m_1$,所以 $G + m_1 > 0$。
\end{itemize}
因此 $v_1$ 为正值。我们需要验证求出的 $v_1$ 是否满足前提 $v_1 > E$:
\begin{align*}
    \frac{m_1 - m_0}{m_1 + G} E > E &\iff \frac{m_1 - m_0}{m_1 + G} > 1 \\
    &\iff m_1 - m_0 > m_1 + G \\
    &\iff -m_0 > G \\
    &\iff m_0 < -G
\end{align*}
该结果满足题目给定条件 $m_0 < -G$,故解有效。
此时 $v_2=0$,代入 (b) 式得 $i_L = G v_1 = G \frac{m_1 - m_0}{m_1 + G} E$。
\textbf{工作点 $Q_+$}:$(\frac{m_1 - m_0}{m_1 + G} E, 0, G \frac{m_1 - m_0}{m_1 + G} E)$。

\textbf{情况 3:$v_1 < -E$}
根据对称性,必然存在另一个工作点 $Q_-$。
\[ v_1 = -\frac{m_1 - m_0}{m_1 + G} E \]
同理可证该解满足 $v_1 < -E$。
\textbf{工作点 $Q_-$}:$(-\frac{m_1 - m_0}{m_1 + G} E, 0, -G \frac{m_1 - m_0}{m_1 + G} E)$。

\section{结果讨论}
\begin{itemize}
    \item 计算结果表明,在 $m_0 < -G$ 条件下,电路存在 \textbf{3个静态工作点}:原点 $Q_0$ 和两个对称的非零工作点 $Q_+, Q_-$。
    \item 结合实验现象分析,系统的轨迹围绕这三个工作点进行复杂的运动,最终在特定的参数下形成了双涡卷(Double Scroll)混沌吸引子。
    \item 本题通过建立蔡氏电路的状态方程,并利用非线性电阻的分段线性特性求解了电路的静态工作点,展示了如何利用基础电路理论工具研究非线性动力学问题。
\end{itemize}

\end{document}